\documentclass[11pt,preprint, authoryear]{elsarticle}

\usepackage{lmodern}
%%%% My spacing
\usepackage{setspace}
\setstretch{1.2}
\DeclareMathSizes{12}{14}{10}{10}

% Wrap around which gives all figures included the [H] command, or places it "here". This can be tedious to code in Rmarkdown.
\usepackage{float}
\let\origfigure\figure
\let\endorigfigure\endfigure
\renewenvironment{figure}[1][2] {
    \expandafter\origfigure\expandafter[H]
} {
    \endorigfigure
}

\let\origtable\table
\let\endorigtable\endtable
\renewenvironment{table}[1][2] {
    \expandafter\origtable\expandafter[H]
} {
    \endorigtable
}


\usepackage{ifxetex,ifluatex}
\usepackage{fixltx2e} % provides \textsubscript
\ifnum 0\ifxetex 1\fi\ifluatex 1\fi=0 % if pdftex
  \usepackage[T1]{fontenc}
  \usepackage[utf8]{inputenc}
\else % if luatex or xelatex
  \ifxetex
    \usepackage{mathspec}
    \usepackage{xltxtra,xunicode}
  \else
    \usepackage{fontspec}
  \fi
  \defaultfontfeatures{Mapping=tex-text,Scale=MatchLowercase}
  \newcommand{\euro}{€}
\fi

\usepackage{amssymb, amsmath, amsthm, amsfonts}

\def\bibsection{\section*{References}} %%% Make "References" appear before bibliography


\usepackage[round]{natbib}
\bibliographystyle{plainnat}

\usepackage{longtable}
\usepackage[margin=2.3cm,bottom=2cm,top=2.5cm, includefoot]{geometry}
\usepackage{fancyhdr}
\usepackage[bottom, hang, flushmargin]{footmisc}
\usepackage{graphicx}
\numberwithin{equation}{section}
\numberwithin{figure}{section}
\numberwithin{table}{section}
\setlength{\parindent}{0cm}
\setlength{\parskip}{1.3ex plus 0.5ex minus 0.3ex}
\usepackage{textcomp}
\renewcommand{\headrulewidth}{0.2pt}
\renewcommand{\footrulewidth}{0.3pt}

\usepackage{array}
\newcolumntype{x}[1]{>{\centering\arraybackslash\hspace{0pt}}p{#1}}

%%%%  Remove the "preprint submitted to" part. Don't worry about this either, it just looks better without it:
\makeatletter
\def\ps@pprintTitle{%
  \let\@oddhead\@empty
  \let\@evenhead\@empty
  \let\@oddfoot\@empty
  \let\@evenfoot\@oddfoot
}
\makeatother

 \def\tightlist{} % This allows for subbullets!

\usepackage{hyperref}
\hypersetup{breaklinks=true,
            bookmarks=true,
            colorlinks=true,
            citecolor=blue,
            urlcolor=blue,
            linkcolor=blue,
            pdfborder={0 0 0}}


% The following packages allow huxtable to work:
\usepackage{siunitx}
\usepackage{multirow}
\usepackage{hhline}
\usepackage{calc}
\usepackage{tabularx}
\usepackage{booktabs}
\usepackage{caption}
\usepackage{colortbl}

\urlstyle{same}  % don't use monospace font for urls
\setlength{\parindent}{0pt}
\setlength{\parskip}{6pt plus 2pt minus 1pt}
\setlength{\emergencystretch}{3em}  % prevent overfull lines
\setcounter{secnumdepth}{5}

%%% Use protect on footnotes to avoid problems with footnotes in titles
\let\rmarkdownfootnote\footnote%
\def\footnote{\protect\rmarkdownfootnote}
\IfFileExists{upquote.sty}{\usepackage{upquote}}{}

%%% Include extra packages specified by user
% Insert custom packages here as follows
% \usepackage{tikz}

%%% Hard setting column skips for reports - this ensures greater consistency and control over the length settings in the document.
%% page layout
%% paragraphs
\setlength{\baselineskip}{12pt plus 0pt minus 0pt}
\setlength{\parskip}{12pt plus 0pt minus 0pt}
\setlength{\parindent}{0pt plus 0pt minus 0pt}
%% floats
\setlength{\floatsep}{12pt plus 0 pt minus 0pt}
\setlength{\textfloatsep}{20pt plus 0pt minus 0pt}
\setlength{\intextsep}{14pt plus 0pt minus 0pt}
\setlength{\dbltextfloatsep}{20pt plus 0pt minus 0pt}
\setlength{\dblfloatsep}{14pt plus 0pt minus 0pt}
%% maths
\setlength{\abovedisplayskip}{12pt plus 0pt minus 0pt}
\setlength{\belowdisplayskip}{12pt plus 0pt minus 0pt}
%% lists
\setlength{\topsep}{10pt plus 0pt minus 0pt}
\setlength{\partopsep}{3pt plus 0pt minus 0pt}
\setlength{\itemsep}{5pt plus 0pt minus 0pt}
\setlength{\labelsep}{8mm plus 0mm minus 0mm}
\setlength{\parsep}{\the\parskip}
\setlength{\listparindent}{\the\parindent}
%% verbatim
\setlength{\fboxsep}{5pt plus 0pt minus 0pt}



\begin{document}

\begin{frontmatter}  %

\title{Marinus\_Tutorial}

% Set to FALSE if wanting to remove title (for submission)




\author[Add1]{Marinus Louw}
\ead{marinuslouw@icloud.com}





\address[Add1]{Financial Economtrics Course, Stellenbosch University, South Africa}

\cortext[cor]{Corresponding author: Marinus Louw}

\begin{abstract}
\small{
Abstract t
}
\end{abstract}

\vspace{1cm}

\begin{keyword}
\footnotesize{
GARCH \\ \vspace{0.3cm}
\textit{JEL classification} 
}
\end{keyword}
\vspace{0.5cm}
\end{frontmatter}



%________________________
% Header and Footers
%%%%%%%%%%%%%%%%%%%%%%%%%%%%%%%%%
\pagestyle{fancy}
\chead{}
\rhead{}
\lfoot{}
\rfoot{\footnotesize Page \thepage\\}
\lhead{}
%\rfoot{\footnotesize Page \thepage\ } % "e.g. Page 2"
\cfoot{}

%\setlength\headheight{30pt}
%%%%%%%%%%%%%%%%%%%%%%%%%%%%%%%%%
%________________________

\headsep 35pt % So that header does not go over title




\section{\texorpdfstring{Introduction
\label{Introduction}}{Introduction }}\label{introduction}

The objective of this tutorial is to accumulate experience in working
with Texevier and RMarkdown. The tutorial was written in a README with
accompanying R Scripts in order to neaten the process. The tutorial
answers the following questions: * Create a summary table showing the
first and second moments of the returns of these stocks for the
following periods: + 2006 - 2008 + 2010 - 2013 * Calculate the
unconditional (full sample) correlations between the stocks. * Plot the
univariate GARCH ht processes for each of the series. * Plot the
cumulative returns series of a portfolio that is equally weighted to
each of the stocks - reweighted each year on the last day of June. * See
if including the GARCH11 conditional volatility series of SLM
(ht,Sanlam) improves the GARCH11 model fit of ABSP (interpret the
p-value of the regressor).

\section{\texorpdfstring{Data \label{Data}}{Data }}\label{data}

After loading our findata we inspect the 1st and 2nd moments. From table
\ref{Moments-A} and table \ref{Moments-B} it is evident that our sample
period from 2006-2008 differs greatly from 2010 - 2013. Upon closer
inspection we can see that both the mean and median for the 2nd moment
of 2006-2008 is more than twice that of 2010-2013. This paired with the
maximum 2nd moment value indicate towards considerably more volatility
during the Global Financial Crisis (GFC) period.

Structurally we can gather from the skewness and kurtosis values of the
returns that our period from 2006-2008 have wide tails and fat tails,
compared to 2010-2013 where we find negative skewness, indicating
relatively more positive returns.

\begin{table}[H]
\centering
\scalebox{0.8}{
\begin{tabular}{rrrrrrrrrrrrrr}
  \hline
 & vars & n & mean & sd & median & trimmed & mad & min & max & range & skew & kurtosis & se \\ 
  \hline
Return &   1 & 4585.00 & -0.01 & 2.11 & 0.00 & -0.02 & 1.44 & -12.58 & 12.64 & 25.22 & 0.05 & 3.31 & 0.03 \\ 
  Return\_Sqd &   2 & 4585.00 & 4.47 & 10.30 & 0.94 & 2.18 & 1.38 & 0.00 & 159.77 & 159.77 & 6.18 & 58.96 & 0.15 \\ 
   \hline
\end{tabular}
}
\caption{1st and 2nd Moments (2006-2008) \label{Moments-A}} 
\end{table}\begin{table}[H]
\centering
\scalebox{0.8}{
\begin{tabular}{rrrrrrrrrrrrrr}
  \hline
 & vars & n & mean & sd & median & trimmed & mad & min & max & range & skew & kurtosis & se \\ 
  \hline
Return &   1 & 7000.00 & 0.04 & 1.43 & 0.00 & 0.06 & 1.03 & -36.42 & 6.87 & 43.29 & -2.46 & 61.35 & 0.02 \\ 
  Return\_Sqd &   2 & 7000.00 & 2.06 & 16.33 & 0.48 & 1.00 & 0.70 & 0.00 & 1326.58 & 1326.58 & 76.38 & 6184.91 & 0.20 \\ 
   \hline
\end{tabular}
}
\caption{1st and 2nd Moments (2010-2013) \label{Moments-B}} 
\end{table}\begin{table}[H]
\centering
\begin{tabular}{rrrrrrrr}
  \hline
 & ABSP & BVT & FSR & NBKP & RMH & SBK & SLM \\ 
  \hline
ABSP & 1.00 & 0.02 & 0.01 & 0.18 & 0.05 & 0.04 & 0.04 \\ 
  BVT & 0.02 & 1.00 & 0.50 & 0.04 & 0.48 & 0.50 & 0.49 \\ 
  FSR & 0.01 & 0.50 & 1.00 & 0.01 & 0.76 & 0.71 & 0.51 \\ 
  NBKP & 0.18 & 0.04 & 0.01 & 1.00 & -0.00 & 0.02 & 0.04 \\ 
  RMH & 0.05 & 0.48 & 0.76 & -0.00 & 1.00 & 0.65 & 0.50 \\ 
  SBK & 0.04 & 0.50 & 0.71 & 0.02 & 0.65 & 1.00 & 0.52 \\ 
  SLM & 0.04 & 0.49 & 0.51 & 0.04 & 0.50 & 0.52 & 1.00 \\ 
   \hline
\end{tabular}
\caption{Unconditional Correlations \label{Correlations}} 
\end{table}\begin{figure}[H]

{\centering \includegraphics{Template_files/figure-latex/figure3-1} 

}

\caption{Pairs Panel \label{Pairs}}\label{fig:figure3}
\end{figure}

Table \ref{Correlations} and Figure \ref{Pairs} above convey the
unconditional correlations between the stocks as a table and a pairs
panel.

\section{Results}\label{results}

The following section illustrate the GARCH ht processes of each stock
along with the GARCH11 model for ABSP with a supplementary external
regressor (SLM's conditional volatility).

\ref{GARCH-A} \ref{GARCH-B}

\begin{table}[H]
\centering
\begin{tabular}{rrrrr}
  \hline
 &  Estimate &  Std. Error &  t value & Pr($>$$|$t$|$) \\ 
  \hline
mu & 0.01 & 0.01 & 0.84 & 0.40 \\ 
  ar1 & -0.08 & 0.03 & -2.88 & 0.00 \\ 
  omega & 0.06 & 0.01 & 5.36 & 0.00 \\ 
  alpha1 & 0.19 & 0.03 & 6.22 & 0.00 \\ 
  beta1 & 0.78 & 0.03 & 24.58 & 0.00 \\ 
  gamma1 & -0.14 & 0.03 & -4.79 & 0.00 \\ 
   \hline
\end{tabular}
\caption{GARCH11 \label{GARCH-A}} 
\end{table}\begin{table}[H]
\centering
\begin{tabular}{rrrrr}
  \hline
 &  Estimate &  Std. Error &  t value & Pr($>$$|$t$|$) \\ 
  \hline
mu & 0.01 & 0.01 & 0.85 & 0.39 \\ 
  ar1 & -0.08 & 0.03 & -2.91 & 0.00 \\ 
  omega & 0.05 & 0.01 & 5.12 & 0.00 \\ 
  alpha1 & 0.20 & 0.03 & 6.53 & 0.00 \\ 
  beta1 & 0.76 & 0.03 & 24.98 & 0.00 \\ 
  gamma1 & -0.15 & 0.03 & -4.96 & 0.00 \\ 
  vxreg1 & 0.01 & 0.00 & 2.91 & 0.00 \\ 
   \hline
\end{tabular}
\caption{GARCH11 with SLM external regressor  \label{GARCH-B}} 
\end{table}

According to the work of Tsay (\protect\hyperlink{ref-Tsay1989}{1989}),
blah blah

\section*{References}\label{references}
\addcontentsline{toc}{section}{References}

\hypertarget{refs}{}
\hypertarget{ref-Tsay1989}{}
Tsay, Ruey S. 1989. ``Testing and Modeling Threshold Autoregressive
Processes.'' \emph{Journal of the American Statistical Association} 84
(405). Taylor \& Francis Group: 231--40.

% Force include bibliography in my chosen format:

\bibliographystyle{Tex/Texevier}
\bibliography{Tex/ref}





\end{document}
