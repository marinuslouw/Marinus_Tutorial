\documentclass[11pt,preprint, authoryear]{elsarticle}

\usepackage{lmodern}
%%%% My spacing
\usepackage{setspace}
\setstretch{1.2}
\DeclareMathSizes{12}{14}{10}{10}

% Wrap around which gives all figures included the [H] command, or places it "here". This can be tedious to code in Rmarkdown.
\usepackage{float}
\let\origfigure\figure
\let\endorigfigure\endfigure
\renewenvironment{figure}[1][2] {
    \expandafter\origfigure\expandafter[H]
} {
    \endorigfigure
}

\let\origtable\table
\let\endorigtable\endtable
\renewenvironment{table}[1][2] {
    \expandafter\origtable\expandafter[H]
} {
    \endorigtable
}


\usepackage{ifxetex,ifluatex}
\usepackage{fixltx2e} % provides \textsubscript
\ifnum 0\ifxetex 1\fi\ifluatex 1\fi=0 % if pdftex
  \usepackage[T1]{fontenc}
  \usepackage[utf8]{inputenc}
\else % if luatex or xelatex
  \ifxetex
    \usepackage{mathspec}
    \usepackage{xltxtra,xunicode}
  \else
    \usepackage{fontspec}
  \fi
  \defaultfontfeatures{Mapping=tex-text,Scale=MatchLowercase}
  \newcommand{\euro}{€}
\fi

\usepackage{amssymb, amsmath, amsthm, amsfonts}

\def\bibsection{\section*{References}} %%% Make "References" appear before bibliography


\usepackage[round]{natbib}
\bibliographystyle{plainnat}

\usepackage{longtable}
\usepackage[margin=2.3cm,bottom=2cm,top=2.5cm, includefoot]{geometry}
\usepackage{fancyhdr}
\usepackage[bottom, hang, flushmargin]{footmisc}
\usepackage{graphicx}
\numberwithin{equation}{section}
\numberwithin{figure}{section}
\numberwithin{table}{section}
\setlength{\parindent}{0cm}
\setlength{\parskip}{1.3ex plus 0.5ex minus 0.3ex}
\usepackage{textcomp}
\renewcommand{\headrulewidth}{0.2pt}
\renewcommand{\footrulewidth}{0.3pt}

\usepackage{array}
\newcolumntype{x}[1]{>{\centering\arraybackslash\hspace{0pt}}p{#1}}

%%%%  Remove the "preprint submitted to" part. Don't worry about this either, it just looks better without it:
\makeatletter
\def\ps@pprintTitle{%
  \let\@oddhead\@empty
  \let\@evenhead\@empty
  \let\@oddfoot\@empty
  \let\@evenfoot\@oddfoot
}
\makeatother

 \def\tightlist{} % This allows for subbullets!

\usepackage{hyperref}
\hypersetup{breaklinks=true,
            bookmarks=true,
            colorlinks=true,
            citecolor=blue,
            urlcolor=blue,
            linkcolor=blue,
            pdfborder={0 0 0}}


% The following packages allow huxtable to work:
\usepackage{siunitx}
\usepackage{multirow}
\usepackage{hhline}
\usepackage{calc}
\usepackage{tabularx}
\usepackage{booktabs}
\usepackage{caption}
\usepackage{colortbl}

\urlstyle{same}  % don't use monospace font for urls
\setlength{\parindent}{0pt}
\setlength{\parskip}{6pt plus 2pt minus 1pt}
\setlength{\emergencystretch}{3em}  % prevent overfull lines
\setcounter{secnumdepth}{5}

%%% Use protect on footnotes to avoid problems with footnotes in titles
\let\rmarkdownfootnote\footnote%
\def\footnote{\protect\rmarkdownfootnote}
\IfFileExists{upquote.sty}{\usepackage{upquote}}{}

%%% Include extra packages specified by user
% Insert custom packages here as follows
% \usepackage{tikz}

%%% Hard setting column skips for reports - this ensures greater consistency and control over the length settings in the document.
%% page layout
%% paragraphs
\setlength{\baselineskip}{12pt plus 0pt minus 0pt}
\setlength{\parskip}{12pt plus 0pt minus 0pt}
\setlength{\parindent}{0pt plus 0pt minus 0pt}
%% floats
\setlength{\floatsep}{12pt plus 0 pt minus 0pt}
\setlength{\textfloatsep}{20pt plus 0pt minus 0pt}
\setlength{\intextsep}{14pt plus 0pt minus 0pt}
\setlength{\dbltextfloatsep}{20pt plus 0pt minus 0pt}
\setlength{\dblfloatsep}{14pt plus 0pt minus 0pt}
%% maths
\setlength{\abovedisplayskip}{12pt plus 0pt minus 0pt}
\setlength{\belowdisplayskip}{12pt plus 0pt minus 0pt}
%% lists
\setlength{\topsep}{10pt plus 0pt minus 0pt}
\setlength{\partopsep}{3pt plus 0pt minus 0pt}
\setlength{\itemsep}{5pt plus 0pt minus 0pt}
\setlength{\labelsep}{8mm plus 0mm minus 0mm}
\setlength{\parsep}{\the\parskip}
\setlength{\listparindent}{\the\parindent}
%% verbatim
\setlength{\fboxsep}{5pt plus 0pt minus 0pt}



\begin{document}

\begin{frontmatter}  %

\title{Marinus\_Tutorial}

% Set to FALSE if wanting to remove title (for submission)




\author[Add1]{Marinus Louw}
\ead{marinuslouw@icloud.com}





\address[Add1]{Financial Economtrics Course, Stellenbosch University, South Africa}

\cortext[cor]{Corresponding author: Marinus Louw}

\begin{abstract}
\small{
Abstract to be written here. The abstract should not be too long and
should provide the reader with a good understanding what you are writing
about. Academic papers are not like novels where you keep the reader in
suspense. To be effective in getting others to read your paper, be as
open and concise about your findings here as possible. Ideally, upon
reading your abstract, the reader should feel he / she must read your
paper in entirety.
}
\end{abstract}

\vspace{1cm}

\begin{keyword}
\footnotesize{
GARCH \\ \vspace{0.3cm}
\textit{JEL classification} L250 \sep L100
}
\end{keyword}
\vspace{0.5cm}
\end{frontmatter}



%________________________
% Header and Footers
%%%%%%%%%%%%%%%%%%%%%%%%%%%%%%%%%
\pagestyle{fancy}
\chead{}
\rhead{}
\lfoot{}
\rfoot{\footnotesize Page \thepage\\}
\lhead{}
%\rfoot{\footnotesize Page \thepage\ } % "e.g. Page 2"
\cfoot{}

%\setlength\headheight{30pt}
%%%%%%%%%%%%%%%%%%%%%%%%%%%%%%%%%
%________________________

\headsep 35pt % So that header does not go over title




\section{\texorpdfstring{Introduction
\label{Introduction}}{Introduction }}\label{introduction}

References are to be made as follows: Fama and French
(\protect\hyperlink{ref-fama1997}{1997}, 33) and Grinold and Kahn
(\protect\hyperlink{ref-grinold2000}{2000}). Source the reference code
from scholar.google.com by clicking on ``cite'' below article name. Then
select BibTeX at the bottom of the Cite window, and proceed to copy and
paste this code into your ref.bib file, located in the directory's Tex
folder. Open this file in Rstudio for ease of management, else open it
in your preferred Tex environment. Add and manage your article details
here for simplicity - once saved, it will self-adjust in your paper.

To reference a section, you have to set a label using
``\textbackslash{}label'' in R, and then reference it in-text as e.g.:
section \ref{Data}.

Writing in Rmarkdown is surprizingly easy - see
\href{https://www.rstudio.com/wp-content/uploads/2015/03/rmarkdown-reference.pdf}{this
website} cheatsheet for a summary on writing Rmd writing tips.

\section{\texorpdfstring{Data \label{Data}}{Data }}\label{data}

Discussion of data should be thorough with a table of statistics and
ideally a figure.

In your tempalte folder, you will find a Data and a Code folder. In
order to keep your data files neat, store all of them in your Data
folder. Also, I strongly suggest keeping this Rmd file for writing and
executing commands, not writing out long pieces of data-wrangling. In
the example below, I simply create a ggplot template for scatter plot
consistency. I suggest keeping all your data in a data folder.

To reference the plot above, add a ``\textbackslash{}label'' after the
caption in the chunk heading, as done above. Then reference the plot as
such: As can be seen, figure \ref{Figure1} is excellent. The nice thing
now is that it correctly numbers all your figures (and sections or
tables) and will update if it moves. The links are also dynamic.

I very strongly suggest using ggplot2 (ideally in combination with
dplyr) using the ggtheme package to change the themes of your figures.

Also note the information that I have placed above the chunks in the
code chunks for the figures. You can edit any of these easily - visit
the Rmarkdown webpage for more information.

\section{Methodology}\label{methodology}

\subsection{Subsection}\label{subsection}

Ideally do not overuse subsections. It equates to bad writing.\footnote{This
  is an example of a footnote by the way. Something that should also not
  be overused.}

\subsection{Math section}\label{math-section}

Equations should be written as such:

\begin{align} 
    y_t &= c + B(L) y_{t-1} + e_t   \label{eq2}    \\ \notag 
    e_t &= H_t^{1/2}  z_t ; \quad z_t \sim  N(0,I_N) \quad \& \quad H_t = D_tR_tD_t \\ \notag
        D_t^2 &= {\sigma_{1,t}, \dots, \sigma_{N,t}}   \\ \notag
        \sigma_{i,t}^2 &= \gamma_i+\kappa_{i,t}  v_{i, t-1}^2 +\eta_i  \sigma_{i, t-1}^2, \quad \forall i \\ \notag
        R_{t, i, j} &= {diag(Q_{t, i, j}}^{-1}) . Q_{t, i, j} . diag(Q_{t, i, j}^{-1})  \\ \notag
        Q_{t, i, j} &= (1-\alpha-\beta)  \bar{Q} + \alpha  z_t  z_t'  + \beta  Q_{t, i, j} \notag
\end{align}

If you would like to see the equations as you type in Rmarkdown, use \$
symbols instead (see this for yourself by adjusted the equation):

\[
\beta = \sum_{i = 1}^{\infty}\frac{\alpha^2}{\sigma_{t-1}^2} \\ 
\int_{x = 1}^{\infty}x_{i} = 1
\]

\section{Results}\label{results}

Tables can be included as follows. Use the \emph{xtable} (or kable)
package for tables. Table placement = H implies Latex tries to place the
table Here, and not on a new page (there are, however, very many ways to
skin this cat. Luckily there are many forums online!).

\begin{table}[H]
\centering
\scalebox{0.8}{
\begin{tabular}{rrrrrrrrrrrrrr}
  \hline
 & vars & n & mean & sd & median & trimmed & mad & min & max & range & skew & kurtosis & se \\ 
  \hline
Return &   1 & 4585.00 & -0.01 & 2.11 & 0.00 & -0.02 & 1.44 & -12.58 & 12.64 & 25.22 & 0.05 & 3.31 & 0.03 \\ 
  Return\_Sqd &   2 & 4585.00 & 4.47 & 10.30 & 0.94 & 2.18 & 1.38 & 0.00 & 159.77 & 159.77 & 6.18 & 58.96 & 0.15 \\ 
   \hline
\end{tabular}
}
\caption{1st and 2nd Moments (2006-2008) \label{tab1}} 
\end{table}\begin{table}[H]
\centering
\scalebox{0.8}{
\begin{tabular}{rrrrrrrrrrrrrr}
  \hline
 & vars & n & mean & sd & median & trimmed & mad & min & max & range & skew & kurtosis & se \\ 
  \hline
Return &   1 & 7000.00 & 0.04 & 1.43 & 0.00 & 0.06 & 1.03 & -36.42 & 6.87 & 43.29 & -2.46 & 61.35 & 0.02 \\ 
  Return\_Sqd &   2 & 7000.00 & 2.06 & 16.33 & 0.48 & 1.00 & 0.70 & 0.00 & 1326.58 & 1326.58 & 76.38 & 6184.91 & 0.20 \\ 
   \hline
\end{tabular}
}
\caption{1st and 2nd Moments (2010-2013) \label{tab1}} 
\end{table}\begin{table}[H]
\centering
\begin{tabular}{rrrrrrrr}
  \hline
 & ABSP & BVT & FSR & NBKP & RMH & SBK & SLM \\ 
  \hline
ABSP & 1.00 & 0.02 & 0.01 & 0.18 & 0.05 & 0.04 & 0.04 \\ 
  BVT & 0.02 & 1.00 & 0.50 & 0.04 & 0.48 & 0.50 & 0.49 \\ 
  FSR & 0.01 & 0.50 & 1.00 & 0.01 & 0.76 & 0.71 & 0.51 \\ 
  NBKP & 0.18 & 0.04 & 0.01 & 1.00 & -0.00 & 0.02 & 0.04 \\ 
  RMH & 0.05 & 0.48 & 0.76 & -0.00 & 1.00 & 0.65 & 0.50 \\ 
  SBK & 0.04 & 0.50 & 0.71 & 0.02 & 0.65 & 1.00 & 0.52 \\ 
  SLM & 0.04 & 0.49 & 0.51 & 0.04 & 0.50 & 0.52 & 1.00 \\ 
   \hline
\end{tabular}
\caption{Unconditional Correlations \label{tab1}} 
\end{table}

\begin{figure}[H]

{\centering \includegraphics{Template_files/figure-latex/figure3-1} 

}

\caption{Pairs Panel \label{lit}}\label{fig:figure3}
\end{figure}

\begin{table}[H]
\centering
\begin{tabular}{rrrrr}
  \hline
 &  Estimate &  Std. Error &  t value & Pr($>$$|$t$|$) \\ 
  \hline
mu & 0.01 & 0.01 & 0.84 & 0.40 \\ 
  ar1 & -0.08 & 0.03 & -2.88 & 0.00 \\ 
  omega & 0.06 & 0.01 & 5.36 & 0.00 \\ 
  alpha1 & 0.19 & 0.03 & 6.22 & 0.00 \\ 
  beta1 & 0.78 & 0.03 & 24.58 & 0.00 \\ 
  gamma1 & -0.14 & 0.03 & -4.79 & 0.00 \\ 
   \hline
\end{tabular}
\caption{GARCH11 \label{tab1}} 
\end{table}\begin{table}[H]
\centering
\begin{tabular}{rrrrr}
  \hline
 &  Estimate &  Std. Error &  t value & Pr($>$$|$t$|$) \\ 
  \hline
mu & 0.01 & 0.01 & 0.85 & 0.39 \\ 
  ar1 & -0.08 & 0.03 & -2.91 & 0.00 \\ 
  omega & 0.05 & 0.01 & 5.12 & 0.00 \\ 
  alpha1 & 0.20 & 0.03 & 6.53 & 0.00 \\ 
  beta1 & 0.76 & 0.03 & 24.98 & 0.00 \\ 
  gamma1 & -0.15 & 0.03 & -4.96 & 0.00 \\ 
  vxreg1 & 0.01 & 0.00 & 2.91 & 0.00 \\ 
   \hline
\end{tabular}
\caption{GARCH11 with SLM external regressor  \label{tab1}} 
\end{table}

``` To reference calculations in text, do this: From table \ref{tab1} we
see the average value of mpg is 20.98.

According to the work of Tsay (\protect\hyperlink{ref-Tsay1989}{1989}),
blah blah !

\section{Conclusion}\label{conclusion}

\section*{References}\label{references}
\addcontentsline{toc}{section}{References}

\hypertarget{refs}{}
\hypertarget{ref-fama1997}{}
Fama, Eugene F, and Kenneth R French. 1997. ``Industry Costs of
Equity.'' \emph{Journal of Financial Economics} 43 (2). Elsevier:
153--93.

\hypertarget{ref-grinold2000}{}
Grinold, Richard C, and Ronald N Kahn. 2000. ``Active Portfolio
Management.'' McGraw Hill New York, NY.

\hypertarget{ref-Tsay1989}{}
Tsay, Ruey S. 1989. ``Testing and Modeling Threshold Autoregressive
Processes.'' \emph{Journal of the American Statistical Association} 84
(405). Taylor \& Francis Group: 231--40.

% Force include bibliography in my chosen format:

\bibliographystyle{Tex/Texevier}
\bibliography{Tex/ref}





\end{document}
